\conclusions  %% \conclusions[modified heading if necessary]

Grounded on the similarity between the gravitational potential produced by a volume source under a density variation and the displacement field produced by a volume source in a half-space under a pressure variation, we have presented the analytical solution for the displacement field due to reservoir compaction with arbitrary geometry and under non-uniform  pressure distribution.
Our approach calculates the displacement field due to a reservoir by approximating its 3D pressure distribution through a piecewise constant function defined on a user-specified grid of 3D vertical prisms juxtaposed in the $x-$, $y-$ and $z-$directions.
By using the nucleus of strain as an infinitesimal volume element for each prism, 
we have calculated the displacement solution yielded by each prism through a 3D integration over the prism volume. 
The sum of the displacements produced by the prisms is the final displacement field due to the whole reservoir.
The adopted exact analytical formulae, based on the gravitational field,  to carry out the full integrations and calculate the displacement field due to reservoir compaction are valid expressions either outside or inside the prisms because the implemented expressions make use of  modified arctangent function.
We have demonstrated the use of these exact analytical expressions by applying them to calculate the displacement fields due to cylindrical reservoirs with uniform and non-uniform pressure distributions and to realistic reservoir model of a production oil field in offshore Brazil with arbitrary geometry and under arbitrary pressure distribution.
All the numerical applications produced null stress fields  at the free surface
showing that the condition of null tractions at the free surface has been met. 
We have presented routines written in Python language (Python 3.7.6)  to calculate the displacement fields due to a reservoir with arbitrary shape and non-uniform distribution of pressure changes. 
The numerical applications and figures showing the results in this article were produced in
Jupyter Notebook. 




\codeavailability{
The current version of our code is freely distributed under the BSD 3-clause licence and it is available for download at Zenodo: http://doi.org/XXXXXXXX. 
The latest development version of our code can be freely downloaded from a repository on GitHub (https://github.com/pinga-lab/reservoir-compaction). 
Instructions for running  the current version of our code are also provided on the repository.
The code is still being improved and we encourage the user to work with the latest development version. 
The code was developed as an open-source Python language (Python 3.7.6).
The numerical applications were produced in Jupyter Notebook.  
} %% use this section when having only software code available


\authorcontribution{TEXT} %% this section is mandatory

\competinginterests{The authors declare that they have no conflict of interest.} %% this section is mandatory even if you declare that no competing interests are present

\begin{acknowledgements}
Valeria C.F. Barbosa was supported by fellowships from: CNPq (grant 307135/2014- 4) and FAPERJ (grant E-26/202.582/2019). 
Vanderlei C. Oliveira Jr. was supported by fellowships from: CNPq (grant 308945/2017-4) and FAPERJ (grant E-26/202.729/2018). 
\end{acknowledgements}
